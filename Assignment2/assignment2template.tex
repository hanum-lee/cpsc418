\documentclass[11pt]{article}

\usepackage{verbatim,amsthm,amsmath,amssymb,amsfonts,url}
\usepackage[margin=1in]{geometry}

\parindent 0pt
\parskip 3mm

\theoremstyle{definition}
\newtheorem*{solution}{Solution}

% Some useful commands

\newcommand{\CBCMAC}{\text{CBC-MAC}}
\newcommand{\AHMAC}{\mathrm{AHMAC}}
\newcommand{\PHMAC}{\mathrm{PHMAC}}
\renewcommand{\pmod}[1]{\mbox{\ $(\ensuremath{\operatorname{mod}}\ {#1})$}}
\newcommand{\GF}{\mbox{GF}}
\newcommand{\Gl}{\mbox{GL}}


\begin{document}

\begin{center}
{\bf \Large CPSC 418 / MATH 318 -- Introduction to Cryptography

ASSIGNMENT 2}
\end{center}

\hrule 	

\textbf{Name:} Crypto Wizard (replace by your name)

\textbf{Student ID:} 0000000 (replace by your ID number)

\medskip \hrule

\begin{enumerate}

	\item[] \textbf{Problem 1} ---  Binary polynomial arithmetic, 20 marks
	
	\begin{enumerate}
		\item %a
		
		\begin{enumerate}
			\item %i
			$\{1,x,x+1,x^{2},x^{2}+1,x^{2}+x,x^{2}+x+1,x^{3},x^{3}+1,x^{3}+x,x^{3}+x+1,x^{3}+x^{2},x^{3}+x^{2}+1,x^{3}+x^{2}+x,x^{3}+x^{2}+x+1 \}$
			
			\item %ii
			For polynomials with 0$\leq$degree$\leq$3 When we substitute 1 to x and solve, if the result is 0, then the polynomial is reducible. So compute $f(1)$ for all the polynomials above and the ones resulting 0 is reducible. Which are: \newline
			$\{x+1,x^{2}+1,x^{2}+x,x^{3}+1,x^{3}+x,x^{3}+x^{2},x^{3}+x^{2}+x+1\}$
			
			\item %iii
			All the other polynomials that weren't listed at ii are irreducible. Which are: \newline
			$\{1,x,x^{2},x^{2}+x+1,x^{3},x^{3}+x+1,x^{3}+x^{2}+x\}$
		
		\end{enumerate}
	
		\item %b
			
		\begin{enumerate}
			\item %i
			$p(x) = x^{4}+x+1$ which means $x^{4}+x+1 = 0$ so $x^4=x+1$\newline
			$f(x) = x^{2}+1$,  $g(x) = x^{3}+x^{2}+1$	\newline
			\[f(x)g(x) = (x^2+1)(x^{3}+x^{2}+1)\]
			\[= x^5 + x^4 + x + x^3 + x^2 + 1\]
			\[= x(x+1) + x+1 + x^3 + x^2 + x + 1\]
			\[= x^2 + x + x + 1 + x^3 + x^2 + x + 1\]
			\[=x^3+x \pmod{x^4+x+1} \]
			
			\item %ii
			$p(x) = x^{4}+x+1$ which means $x^{4}+x+1 = 0$ so $x^4+x=1$\newline
			$f(x) = x$ and we have to find a polynomial $g(x)$ that is:
			\[f(x)g(x) = 1 = x^4 + x\] So,
			\[xg(x) = x^4 + x\]
			\[g(x) = x^3 + 1\]
		
		\end{enumerate}
	
		\item %c
		
		\begin{enumerate}
		
			\item %i
			$M(y) = y^4 + 1$ which means $y^4 =1 $
			Suppose there is a arbitrary equation $g(y)$ which $g(y) = ay^3+by^2+cy+d$ where $a,b,c,d \in \mathcal{Z} $ 
			\[g(y) \bullet y  = (ay^3+by^2+cy+d)(y)\]
			\[=ay^4+by^3+cy^2+dy\]
			\[=by^3 + cy^2 +dy + a\]
			for any arbitrary equation. Which shows that the coefficients are circular left shift of the vector by one.
				
			\item %ii
				
			\item %iii
		\end{enumerate}
	\end{enumerate}
	
	
	\item[] \textbf{Problem 2} ---  Arithmetic with the constant polynomial of {\sc MixColumns} in
	    AES, 13 marks
	
	\begin{enumerate}
	
		\item %a
		
		\item %b
		
		\begin{enumerate}
			\item %i
			
			\item %ii
		\end{enumerate}
		
		\item %c
		
		\begin{enumerate}
			\item %i
			
			\item %ii
		\end{enumerate}
	\end{enumerate}
	
	
	\item[] \textbf{Problem 3} --- Error propagation in block cipher modes, 12 marks
	
	\begin{enumerate}
		\item %a
		
		\begin{enumerate}
			\item %i
			
			\item %ii
			
			\item %iii
			
			\item %iv
			
			\item %v
		
		\end{enumerate}
		
		\item %b
	
	\end{enumerate}
	
	
	\item[] \textbf{Problem 4} --- Flawed MAC designs, 24 marks
	
	\begin{enumerate}
	
		\item %a
		
		\begin{enumerate}
			\item %i
			
			\item %ii
		\end{enumerate}
		
		\item %b
		
		\begin{enumerate}
		
			\item %i
			
			\item %ii
		
		\end{enumerate}
	\end{enumerate}
	
	
	\item[] \textbf{Problem 5} --- Cryptanalysis of a class of linear ciphers, 31 marks
	
	
	\begin{enumerate}
	\item %a
	
	\item %b
	
	\item %c
	
	\item %d
	
	\item %e
	
	\item %f
	
	\item %g
	
	\item %h
	\end{enumerate}
	
	
	\item[] \textbf{Problem 7} --- Playfair cipher cryptanalysis, 10 marks]

\end{enumerate}

\end{document}

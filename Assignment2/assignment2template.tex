\documentclass[11pt]{article}

\usepackage{verbatim,amsthm,amsmath,amssymb,amsfonts,url}
\usepackage[margin=1in]{geometry}

\parindent 0pt
\parskip 3mm

\theoremstyle{definition}
\newtheorem*{solution}{Solution}

% Some useful commands

\newcommand{\CBCMAC}{\text{CBC-MAC}}
\newcommand{\AHMAC}{\mathrm{AHMAC}}
\newcommand{\PHMAC}{\mathrm{PHMAC}}
\renewcommand{\pmod}[1]{\mbox{\ $(\ensuremath{\operatorname{mod}}\ {#1})$}}
\newcommand{\GF}{\mbox{GF}}
\newcommand{\Gl}{\mbox{GL}}


\begin{document}

\begin{center}
{\bf \Large CPSC 418 / MATH 318 -- Introduction to Cryptography

ASSIGNMENT 2}
\end{center}

\hrule 	

\textbf{Name:} Hanum Lee

\textbf{Student ID:} 30010205

\medskip \hrule

\begin{enumerate}

	\item[] \textbf{Problem 1} ---  Binary polynomial arithmetic, 20 marks
	
	\begin{enumerate}
		\item %a
		
		\begin{enumerate}
			\item %i
			$\{1,x,x+1,x^{2},x^{2}+1,x^{2}+x,x^{2}+x+1,x^{3},x^{3}+1,x^{3}+x,x^{3}+x+1,x^{3}+x^{2},x^{3}+x^{2}+1,x^{3}+x^{2}+x,x^{3}+x^{2}+x+1 \}$
			
			\item %ii
			For polynomials with 0$\leq$degree$\leq$3 When we substitute 1 to x and solve, if the result is 0, then the polynomial is reducible. So compute $f(1)$ for all the polynomials above and the ones resulting 0 is reducible. Which are: \newline
			$\{x+1,x^{2}+1,x^{2}+x,x^{3}+1,x^{3}+x,x^{3}+x^{2},x^{3}+x^{2}+x+1\}$
			
			\item %iii
			All the other polynomials that weren't listed at ii are irreducible. Which are: \newline
			$\{1,x,x^{2},x^{2}+x+1,x^{3},x^{3}+x+1,x^{3}+x^{2}+x\}$
		
		\end{enumerate}
	
		\item %b
			
		\begin{enumerate}
			\item %i
			$p(x) = x^{4}+x+1$ which means $x^{4}+x+1 = 0$ so $x^4=x+1$\newline
			$f(x) = x^{2}+1$,  $g(x) = x^{3}+x^{2}+1$	\newline
			\[f(x)g(x) = (x^2+1)(x^{3}+x^{2}+1)\]
			\[= x^5 + x^4 + x + x^3 + x^2 + 1\]
			\[= x(x+1) + x+1 + x^3 + x^2 + x + 1\]
			\[= x^2 + x + x + 1 + x^3 + x^2 + x + 1\]
			\[=x^3+x \pmod{x^4+x+1} \]
			
			\item %ii
			$p(x) = x^{4}+x+1$ which means $x^{4}+x+1 = 0$ so $x^4+x=1$\newline
			$f(x) = x$ and we have to find a polynomial $g(x)$ that is:
			\[f(x)g(x) = 1 = x^4 + x\] So,
			\[xg(x) = x^4 + x\]
			\[g(x) = x^3 + 1\]
		
		\end{enumerate}
	
		\item %c
		
		\begin{enumerate}
		
			\item %i
			$M(y) = y^4 + 1$ which means $y^4 =1 $
			Suppose there is a arbitrary equation $g(y)$ which $g(y) = ay^3+by^2+cy+d$ where $a,b,c,d \in \mathcal{Z} $ 
			\[g(y) \bullet y  = (ay^3+by^2+cy+d)(y)\]
			\[=ay^4+by^3+cy^2+dy\]
			\[=by^3 + cy^2 +dy + a\]
			for any arbitrary equation. Which shows that the coefficients are circular left shift of the vector by one.
				
			\item %ii
				Since $M(y) = y^4 + 1$, that means $y^4 =1 $. Since the degree of $y^4 =1 $ is 0,the calculation of degree is equivalent to $i$ modular $4$.So when $i$ in $y^i$ is $0 \leq i$, $i \equiv j \mod4 $ where $0 \leq j \leq 3 $ is true.
			\item %iii
				With proof of c.ii, we can conclude that $y^i = y^j$ where $i \equiv j \mod 4$ is true. With proof of  c.i, we can conclude that multiplying $y$ to the any arbitrary polynomial results in circular left shift of the vector by one. So if we combine both proof, by treating multiplying $y^i$ to any arbitrary polynomial as multiplying $y$, $j$ times to polynomial where $i \equiv j \mod 4$,we can conclude that multiplication of $y^i$ is equal to circular left shit of the vector by $j$ bytes.
		\end{enumerate}
	\end{enumerate}
	
	\newpage
	\item[] \textbf{Problem 2} ---  Arithmetic with the constant polynomial of {\sc MixColumns} in
	    AES, 13 marks
	
	\begin{enumerate}
	
		\item %a
			\[c(y)=(03)y^3 + (01)y^2 + (01)y + 02\]
			Converting hex into bit,
			\[(00000011)y^3+(00000001)y^2+(00000001)y+(00000010)\]
			Then convert each byte into corresponding polynomials,
			\[(x+1)y^3+(1)y^2+(1)y+(x)\]
		\item %b
		
		\begin{enumerate}
			\item %i
				We can express $b$ as polynomial:$b_7x^7+b_6x^6+b_5x^5+b_4x^4+b_3x^3+b_2x^2+b_1x+b_0$. Also we can express $x^8 = x^4+x^3=x+1$ and $(02)$ as $x$
				\[b(02) = (b_7x^7+b_6x^6+b_5x^5+b_4x^4+b_3x^3+b_2x^2+b_1x+b_0)(x)\]
				\[=b_7x^8+b_6x^7+b_5x^6+b_4x^5+b_3x^4+b_2x^3+b_1x^2+b_1x\]
				\[=b_7(x^4+x^3+x+1) + b_6x^7+b_5x^6+b_4x^5+b_3x^4+b_2x^3+b_1x^2+b_1x\]
				\[ = b_7x^4 +b_7x^3+b_7x+ b_7+ b_6x^7+b_5x^6+b_4x^5+b_3x^4+b_2x^3+b_1x^2+b_1x\]
				\[= b_6x^7+b_5x^6+b_4x^5+(b_3+b_7)x^4+(b_2+b_7)x^3+b_1x^2+(b_1+b_7)x + b_7\]
			\item %ii
				We can express $(03)$ as $x+1$. We can express $b(03)$ as $b(02) + b$.
				\[b(02) + b = \]
				\[(b_6x^7+b_5x^6+b_4x^5+(b_3+b_7)x^4+(b_2+b_7)x^3+b_1x^2+(b_1+b_7)x + b_7)\]  \[ + (b_7x^7+b_6x^6+b_5x^5+b_4x^4+b_3x^3+b_2x^2+b_1x+b_0)\]
				\[= (b_6+b_7)x^7+(b_5+b_6)x^6+(b_4+b_5)x^5+(b_3+b_7+b_4)x^4+(b_2+b_7+b_3)x^3\]
				\[+(b_1+b_2)x^2+(b_1+b_1+b_7)x+(b_0+b_7)\]
				\[= (b_6+b_7)x^7+(b_5+b_6)x^6+(b_4+b_5)x^5+(b_3+b_7+b_4)x^4+(b_2+b_7+b_3)x^3\]
				\[+(b_1+b_2)x^2+(b_1+b_7)x+(b_0+b_7)\]
		\end{enumerate}
		
		\item %c
		
		\begin{enumerate}
			\item %i
			\[((03)y^3+(01)y^2+(01)y+(02))(s_3y^3+s_2y^2+s_1y+s_0) \mod{y^4 + 1}\]
			\[(03)(s_3)y^6+(03)(s_2)y^5+(03)(s_1)y^4+(03)(s_0)y^3+s_3y^5+s_2y^4+s_1y^3+s_0y^2+s_3y^4+ s_2y^3+s_1y^2+s_0y\]
			\[+(02)(s_3)y^3+(02)(s_2)y^2+(02)(s_1)y+(02)(s_0)\]
			\[= (03)(s_3)y^6+((03)(s_2) + s_3)y^5 + ((03)(s_1)+s_2+s_3)y^4+((03)(s_0)+s_1+s_2+(02)(s_3))y^3\] 
			\[+ (s_0+s_1+(02)(s_2))y^2 + (s_0+(02)(s_1))y + (02)(s_0)\]
			\[= (03)(s_3)y^2+((03)(s_2) + s_3)y + ((03)(s_1)+s_2+s_3)+((03)(s_0)+s_1+s_2+(02)(s_3))y^3\] 
			\[+ (s_0+s_1+(02)(s_2))y^2 + (s_0+(02)(s_1))y + (02)(s_0)\]
			\[=((03)(s_0)+s_1+s_2+(02)(s_3))y^3+ (s_0+s_1+(02)(s_2)+03(s_3)y^2\]
			\[ + ((02)(s_3)+s_2+(03)(s_1)+(01)(s_0))y + ((02)(s_0)+(03)(s_1)+s_2+s_3)\]
			\item %ii
				\[C = 
				\begin{bmatrix}
					02 & 03 & 01 & 01 \\
					02 & 01 & 03 & 04\\
					01 &01 &02 &03 \\
					03 & 01 & 01 & 02\\
				\end{bmatrix}\]
		\end{enumerate}
	\end{enumerate}
	
	\newpage
	\item[] \textbf{Problem 3} --- Error propagation in block cipher modes, 12 marks
	
	\begin{enumerate}
		\item %a
		
		\begin{enumerate}
			\item %i
			Only the $M_i$ block will be affected. ECB decrypts each block individually.
			\item %ii
			Only the $M_i,M_{i+1}$ blocks will be affected. CBC decrypts each block by the cipher text XOR with previous output.
			\item %iii
			Only the $M_i$ block will be affected. OFB decrypts the initialization vector first then XOR the cipher block.
			\item %iv
			Only the $M_i,M_{i+1}$ blocks will be affected. CFB XOR the cipher block with IV then use the cipher block as the IV for next decryption. 
			\item %v
			Only the $M_i$ block will be affected. CTR decrypts each cipher individual with incrementing IV.
		
		\end{enumerate}
		
		\item %b
			All of the blocks after $M_i$ will be affected. Since in CBC, they feed in the cipher text block to encrypt next cipher text block. All of the encryption after $M_i$ will be affected.
	
	\end{enumerate}
	
	\newpage
	\item[] \textbf{Problem 4} --- Flawed MAC designs, 24 marks
	
	\begin{enumerate}
	
		\item %a
		
		\begin{enumerate}
			\item %i
				Attacker can computer PHMAC$_K(M_2)$ without the knowledge of $K$. Since $M_2$ is $M_1||X$, it is same as having an input of $M = P_1||P_2||...||P_L||X_1||X_2||...||X_L$. And Since we already know PHMAC$_K$, in 2nd step of algorithm, we treat H as PHMAC$_K$ and $P_i$ as $X_i$ then run the algorithm.
			\item %ii
			(Idea from: https://crypto.stackexchange.com/questions/11670/why-is-appending-the-key-to-a-mesage-and-then-hashing-that-insecure-if-the-hash)\newline
			If you are appending key at last, that means It is same thing as computing $ITHASH(M)$ then at line 2 of the algorithm, use it as $H$ and use $K$ as $P_i$. Since ITHASH is not weakly collision resistance. We can find collision $M'$ on $M$ such that $ITHASH(M) = ITHASH(M')$. Then we can find $ITHASH(M'||K)$ with a cost of $\frac{n}{2}$.
		\end{enumerate}
		
		\item %b
		
		\begin{enumerate}
		
			\item %i
				It is easy to find CBC-MAC$_K$ of $M_3$ without knowledge of $K$. CBC-MAC$_K (M_2)$ is same as $E_K$(CBC-MAC$_K (M_1)$) since for the first loop of algorithm, $C = 0^n$. We can rewrite CBC-MAC$_K (M_3)$ as $E_K($CBC-MAC$(M_1))$ since for the first loop of the algorithm $C = 0^n$ then on second run, $P_i = 0^n$. Therefore, CBC-MAC$_K(M_3) = $  CBC-MAC$_K(M_2)$.
			\item %ii
				It is possible to find CBC-MAC$_K(M_4)$ without knowledge of $K$ if we have pairs of messages and hash values. We can expend CBC-MAC$_K(M_4)$ as 
				\[E_K(M_2) \oplus E_K(M_1) \oplus E_K(M_2) \oplus X \]
				With the property of XOR, we can cancel out $E_K(M_2)$ so CBC-MAC$_K(M_4)$ can be expended as
				\[E_K(M_1) \oplus X\]
				Which is same as CBC-MAC$_K(M_3)$. Since we know ($M_3$,CBC-MAC$_K(M_3)$), we can figure out CBC-MAC$_K(M_4)$
		\end{enumerate}
	\end{enumerate}
	
	
	
	
	%\item[] \textbf{Problem 7} --- Playfair cipher cryptanalysis, 10 marks]

\end{enumerate}

\end{document}

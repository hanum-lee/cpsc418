\documentclass[11pt]{article}

\usepackage{verbatim,amsthm,amsmath,amssymb,amsfonts,url}
\usepackage[margin=1in]{geometry}

\parindent 0pt
\parskip 3mm

\theoremstyle{definition}
\newtheorem*{solution}{Solution}

% Some useful commands

\newcommand{\CBCMAC}{\text{CBC-MAC}}
\newcommand{\AHMAC}{\mathrm{AHMAC}}
\newcommand{\PHMAC}{\mathrm{PHMAC}}
\renewcommand{\pmod}[1]{\mbox{\ $(\ensuremath{\operatorname{mod}}\ {#1})$}}
\newcommand{\GF}{\mbox{GF}}
\newcommand{\Gl}{\mbox{GL}}


\begin{document}

\begin{center}
{\bf \Large CPSC 418 / MATH 318 -- Introduction to Cryptography

ASSIGNMENT 2}
\end{center}

\hrule 	

\textbf{Name:} Crypto Wizard (replace by your name)

\textbf{Student ID:} 0000000 (replace by your ID number)

\medskip \hrule

\begin{enumerate}

	\item[] \textbf{Problem 1} ---  Binary polynomial arithmetic, 20 marks
	
	\begin{enumerate}
		\item %a
		
		\begin{enumerate}
			\item %i
			
			\item %ii
			
			\item %iii
		
		\end{enumerate}
	
		\item %b
			
		\begin{enumerate}
			\item %i
			
			\item %ii
		
		\end{enumerate}
	
		\item %c
		
		\begin{enumerate}
		
		\item %i
		
		\item %ii
		
		\item %iii
		\end{enumerate}
	\end{enumerate}
	
	
	\item[] \textbf{Problem 2} ---  Arithmetic with the constant polynomial of {\sc MixColumns} in
	    AES, 13 marks
	
	\begin{enumerate}
	
		\item %a
		
		\item %b
		
		\begin{enumerate}
			\item %i
			
			\item %ii
		\end{enumerate}
		
		\item %c
		
		\begin{enumerate}
			\item %i
			
			\item %ii
		\end{enumerate}
	\end{enumerate}
	
	
	\item[] \textbf{Problem 3} --- Error propagation in block cipher modes, 12 marks
	
	\begin{enumerate}
		\item %a
		
		\begin{enumerate}
			\item %i
			
			\item %ii
			
			\item %iii
			
			\item %iv
			
			\item %v
		
		\end{enumerate}
		
		\item %b
	
	\end{enumerate}
	
	
	\item[] \textbf{Problem 4} --- Flawed MAC designs, 24 marks
	
	\begin{enumerate}
	
		\item %a
		
		\begin{enumerate}
			\item %i
			
			\item %ii
		\end{enumerate}
		
		\item %b
		
		\begin{enumerate}
		
			\item %i
			
			\item %ii
		
		\end{enumerate}
	\end{enumerate}
	
	
	\item[] \textbf{Problem 5} --- Cryptanalysis of a class of linear ciphers, 31 marks
	
	
	\begin{enumerate}
	\item %a
	
	\item %b
	
	\item %c
	
	\item %d
	
	\item %e
	
	\item %f
	
	\item %g
	
	\item %h
	\end{enumerate}
	
	
	\item[] \textbf{Problem 7} --- Playfair cipher cryptanalysis, 10 marks]

\end{enumerate}

\end{document}

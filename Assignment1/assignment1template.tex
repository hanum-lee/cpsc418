\documentclass[11pt]{article}

\usepackage{fullpage, verbatim,amsthm,amsmath,amssymb,amsfonts}

\parindent 0pt
\parskip 3mm

\theoremstyle{definition}
\newtheorem*{solution}{Solution}

\begin{document}

	\begin{center}
		{\bf \Large CPSC 418 / MATH 318 --- Introduction to Cryptography
		
		ASSIGNMENT 1 %--- SOLUTION KEY
		}
	\end{center}
	
	\hrule 	
	
	\textbf{Name:} test  \\
	\textbf{Student ID:} test 
	
	\medskip \hrule
	
	\begin{enumerate}
	
		\item[] \textbf{Problem 1} --- Superencipherment for substitution ciphers, 12 marks
	
	\begin{enumerate}
		\item
		\begin{enumerate}
			\item Single Cipher: $E_{k}(M) \equiv M + K \pmod{26}$\\
				Double Cipher:\\
				\[E_{K_{1}}(E_{K_{2}}(M)) \equiv E_{K_{1}}(M+K_{2}\pmod{26})\]
				\[= M + K_{2}\pmod{26} + K_{1}\pmod{26}\]
				\[= M + K_{3}\pmod{26}\]
				for $K_{1,2,3} \in Keyspace$
			\item Assume: $E_{k}(m) \equiv m + k\pmod{26}$ For all $m \in M \& k \in K$\\
			Base case: $E_{K_{1}}(E_{K_{2}}(M)) \equiv  M + K_{3}\pmod{26}$ (Proven at 1.(a).i)\\
			Inductive Hypothesis: $E_{k_{n}}(E_{k_{n-1}}...E_{k_{1}}(m)) \equiv m + (k_{n+n-1+...+1} ) \pmod{26}$ is true for $n > 2 $\\
			Inductive Step:\\
			Want to show that: $E_{k_{n+1}}(E_{k_{n}}...E_{k_{1}}(m)) \equiv m + k_{(n+1)+n+..+1} \pmod{26}$\\
			$E_{k_{n+1}}(E_{k_{n}}...E_{k_{1}}(m)) = E_{k_{k+1}}(m+k_{n+...+1}\pmod{26})$ (with Inductive Hypothesis)
			\[\equiv m + k_{(n+1) + n + .. +1} \pmod{26} \] where $k_{(n+1) + n+...+1} = k_q$ for $q \in Z $\\
			Therefore, the induction holds true.
		\end{enumerate}
		\item
		
	\end{enumerate}
	
	
		\item[] \textbf{Problem 2} --- Key size versus password size, 21 marks]
	
	
	
	\begin{enumerate}
		\item $2^{7}*2^{7}*2^{7}*2^{7}*2^{7}*2^{7}*2^{7}2^{7} = 2^{7*8} = 2^{56}$
		\item
		
		\begin{enumerate}
			\item $98*98*98*98*98*98*98*98 = 98^{8}$
			\item$\dfrac{98^8}{2^{56}} * 100 = 11.81\%$
		\end{enumerate}
		
		\item $H(X) = {\sum}_{i=0}^{n} p(X_{i}) \log_{2} \dfrac{1}{p(X_{i})}$\\
		In this case:\\
		$H(X) = 8 * {\sum}_{i=1}^{n} \dfrac{1}{n} \log_{2} n$ (Since all characters have equal chance of appearing for each character and there are 8 characters in passwords )\\
		$= 8 * \log_{2} 94$\\
		$ = 52.43$
		
		\item Similar as above\\
		$H(X) = 8 * {\sum}_{i=1}^{n} \dfrac{1}{n} \log_{2} n$\\
		$= 8 * \log_{2} 26$\\
		$= 37.60$

		\item
		\begin{enumerate}
			\item $128 = l * \log_{2} 94 $ where $l$ is length of the password\\ 
			$l = \dfrac{128}{\log_{2} 94}$\\
			$l = 19.35$\\
			So, at least 20 characters
			
			\item $128 = l * \log_{2} 26 $ where $l$ is length of the password\\
			$l = \dfrac{128}{\log_{2} 26}$\\
			$l = 27.23$\\
			So, at least 28 characters
			
		\end{enumerate}
	\end{enumerate}
	
	
	
	\item[] \textbf{Problem 3} --- Equiprobability maximizes entropy for two outcomes, 12 marks
	
	
	\begin{enumerate}
		\item 
		$H(X) = p(X_{1}) \log_{2}(\dfrac{1}{p(X_{1})}) + p(X_{2})\log_{2}(\dfrac{1}{p(X_{2})})$\\
		$= \dfrac{1}{4} \log_{2} 4 + \dfrac{3}{4} \log_{2} \dfrac{4}{3}$\\
		$= \dfrac{1}{2} + 0.311$\\
		$= 0.81$
		\item To find maximum of a function, first, we need to find derivative of the function.\\
		$\dfrac{d}{dy} -p \log_{2}(p)-(1-p)log_{2}(1-p)$\\
		$ =- \dfrac{d}{dy} (p) \dfrac{\ln p}{\ln 2} - (1-p)\dfrac{\ln (1-p)}{\ln 2} $
		$ = $
		
		
		
		\item  Find where the value of the derivative above is 0. Then find which point goes from negative to positive. 
	
	\end{enumerate}
	
	\item[] \textbf{Problem 4} --- Conditional entropy, 12 marks
		
		\begin{enumerate}
			\item
			
			\item
			\item
		\end{enumerate}
	\end{enumerate}
	
	
	\begin{enumerate}
	\item[] \textbf{Problem 5} --- Perfect secrecy and joint entropy, 43 marks
	
	\textbf{*** Remove the text for this problem if you don't attempt it. ***}
	
	\begin{enumerate}
	
	\item
	\begin{enumerate}
	
	\item
	
	\item
	
	\item
	
	\item
	
	\item
	\end{enumerate}
	
	\item
	\begin{enumerate}
	\item
	
	\item
	
	\item
	
	\item
	
	\item
	
	\item
	
	\item
	\end{enumerate}
	\end{enumerate}
	
	\item[] \textbf{Problem 7} --- Mixed Vigen\`ere cipher cryptanalysis, 10 marks
	
	\textbf{*** Remove the text for this problem if you don't attempt it. ***}
	
	
	\end{enumerate}

\end{document}
